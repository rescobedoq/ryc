\documentclass[12pt,a4paper]{article}
\usepackage[spanish]{babel}
\usepackage[utf8]{inputenc}
\usepackage[T1]{fontenc}
\usepackage{geometry}
\geometry{margin=2.2cm}
\usepackage{parskip}
\usepackage{graphicx}
\usepackage{hyperref}

\begin{document}

\begin{titlepage}
    \centering
    {\large Universidad Nacional de San Agustín de Arequipa\\[0.2cm]
    Facultad de Ingeniería de Producción y Servicios\\[0.2cm]
    Escuela Profesional de Ingeniería de Sistemas\\[0.4cm]}
    {\large Curso: Tecnología de Objetos\\[0.2cm]}
    {\large Docente: Ing. Richart Smith Escobedo Quispe\\[1cm]}

    {\LARGE \textbf{Memoria Descriptiva}\\[0.5cm]}
    {\Large Videojuego Educativo de Carreras en Unity\\[0.2cm]
    \Large \textit{“Rápidos y Curiosos”}\\[1.2cm]}

    \begin{flushleft}
    \textbf{Integrantes del equipo:}
    \begin{itemize}
        \item Choquehuanca Zapana Hernan Andy (Rama: hz07s/workspace)
        \item Larico Rodriguez Bryan Fernando (Rama: BryanLarico/workspace)
        \item Portugal Portugal Eduardo Sebastian (Rama: Eduardo-P/workspace)
        \item Mamani Anahua Victor Narciso (Rama: victorma18/workspace)
    \end{itemize}
    \end{flushleft}

    \vfill
    {\large Arequipa, 7 de diciembre de 2024}
\end{titlepage}

\tableofcontents
\newpage

\section{Introducción}

La presente memoria descriptiva documenta el diseño, desarrollo e implementación del videojuego educativo de carreras \textit{“Rápidos y Curiosos”}, creado en el motor Unity como proyecto del curso Tecnología de Objetos. El sistema combina mecánicas de conducción en tiempo real con un componente de preguntas y respuestas que introduce contenidos educativos durante la experiencia de juego.

El jugador recorre un circuito y, en puntos clave de la pista, se activan preguntas con varias alternativas representadas por portales interactivos. Al atravesar el portal que corresponde a la respuesta correcta, el vehículo recibe un impulso temporal de velocidad; en caso de error, se aplica una penalización que reduce el desempeño por algunos segundos, reforzando así la retroalimentación inmediata y la atención sostenida del usuario.

\subsection{Contexto y motivación}

Los videojuegos educativos o \textit{serious games} se han consolidado como herramientas de apoyo al proceso de enseñanza–aprendizaje, al integrar dinámicas lúdicas con objetivos formativos concretos. En este contexto, \textit{“Rápidos y Curiosos”} busca aprovechar la familiaridad de los estudiantes con los juegos de carreras para reforzar contenidos académicos mediante la toma de decisiones rápidas.

El proyecto se plantea además como un ejercicio integral de aplicación de programación orientada a objetos, uso de patrones de diseño y trabajo colaborativo con control de versiones, alineado con los lineamientos del curso y la rúbrica de evaluación.

\subsection{Problema y oportunidad}

En entornos educativos tradicionales resulta difícil mantener la atención constante del estudiante durante actividades teóricas o evaluaciones repetitivas. Asimismo, muchos recursos digitales de evaluación carecen de elementos de motivación extrínseca más allá de la calificación numérica.

El videojuego propone una alternativa en la que el rendimiento académico (responder correctamente) se vincula de forma directa con una ventaja observable en el juego (mayor velocidad y mejor tiempo de carrera), generando una relación inmediata entre aprendizaje y desempeño lúdico.

\subsection{Objetivos}

\subsubsection{Objetivo general}

Desarrollar un videojuego educativo de carreras en Unity que integre un sistema de preguntas y respuestas a través de portales interactivos, promoviendo el aprendizaje mediante la toma de decisiones rápidas y la retroalimentación inmediata.

\subsubsection{Objetivos específicos}

\begin{itemize}
    \item Implementar una mecánica básica de conducción (aceleración, frenado y giro) adecuada para un circuito de carreras educativo.
    \item Diseñar e integrar un sistema de preguntas con múltiples opciones representadas por portales en la pista.
    \item Asociar respuestas correctas e incorrectas con efectos de impulso y penalización de velocidad sobre el vehículo.
    \item Incorporar al menos tres patrones de diseño en la arquitectura del software (por ejemplo: Singleton, State y Observer).
    \item Proporcionar un modo de juego para uno o varios jugadores locales, con selección de vehículos y visualización de estadísticas.
\end{itemize}

\subsection{Alcance y limitaciones}

El alcance de la versión actual incluye:

\begin{itemize}
    \item Una pista jugable con zonas de activación de preguntas.
    \item Sistema de preguntas de opción múltiple integrado con portales en el circuito.
    \item Modos de juego para un jugador y multijugador local en un mismo equipo.
    \item Selección de vehículos con diferentes atributos de manejo.
    \item Interfaz gráfica de usuario orientada a escritorio (Windows).
\end{itemize}

Entre las principales limitaciones se encuentran:

\begin{itemize}
    \item El sistema de audio (música y efectos de sonido) no se completó según el cronograma inicial y se considera trabajo futuro.
    \item La gestión de bancos de preguntas se realiza dentro del proyecto (código o recursos de Unity) y aún no se expone un panel avanzado para docentes.
    \item El juego está pensado para PC con teclado o mando; no se contempla por ahora una versión móvil.
\end{itemize}

\section{Marco teórico y tecnológico}

\subsection{Videojuegos educativos y toma de decisiones}

El videojuego se enmarca dentro de los \textit{juegos serios}, donde la dimensión lúdica se utiliza como medio para lograr objetivos educativos. En este caso, la mecánica de elegir portales bajo presión de tiempo refuerza habilidades de concentración, memoria y toma de decisiones, además de los contenidos específicos que se evalúan.

\subsection{Motor Unity y programación orientada a objetos}

Unity se utilizó como plataforma principal de desarrollo, empleando C\# como lenguaje de programación orientado a objetos. Esta elección permite estructurar el código en clases y componentes reutilizables (controladores de vehículo, gestores de preguntas, controladores de escena, etc.), facilitando la extensión del proyecto en iteraciones futuras.

\subsection{Patrones de diseño}

Con el fin de mejorar la mantenibilidad y escalabilidad del sistema, se aplicaron diversos patrones de diseño en los componentes principales:

\begin{itemize}
    \item \textbf{Singleton}: para gestores globales como el administrador de juego y el gestor de preguntas, garantizando una única instancia accesible desde distintas escenas.
    \item \textbf{State}: para modelar los estados del vehículo (normal, impulsado, penalizado), separando la lógica de comportamiento según el estado actual.
    \item \textbf{Observer}: para notificar a la interfaz de usuario y a los portales cuando se genera una nueva pregunta o se registra una respuesta.
\end{itemize}

\section{Análisis y diseño del sistema}

\subsection{Requisitos funcionales}

A nivel general, el sistema cumple con los siguientes requisitos funcionales principales:

\begin{itemize}
    \item RF01: Permitir la conducción de un vehículo a través de un circuito de carreras.
    \item RF02: Mostrar preguntas educativas durante la carrera con varias opciones de respuesta.
    \item RF03: Asociar cada opción con un portal interactivo en la pista.
    \item RF04: Aplicar impulso de velocidad al vehículo cuando la respuesta es correcta.
    \item RF05: Aplicar penalización de velocidad cuando la respuesta es incorrecta.
    \item RF06: Ofrecer un menú principal con selección de modo de juego y de vehículo.
    \item RF07: Mostrar información relevante en pantalla (velocidad, vuelta, mensajes de acierto/error).
\end{itemize}

\subsection{Requisitos no funcionales}

Entre los requisitos no funcionales más relevantes se incluyen:

\begin{itemize}
    \item RNF01: Interfaz intuitiva que permita aprender a jugar en pocos minutos.
    \item RNF02: Rendimiento adecuado para mantener una tasa de fotogramas estable en equipos de gama media.
    \item RNF03: Código organizado en clases y scripts coherentes, aplicando patrones de diseño y buenas prácticas.
    \item RNF04: Uso de herramientas de software libre y licencias adecuadas para los recursos empleados.
\end{itemize}

\subsection{Arquitectura general}

La arquitectura se organiza alrededor de escenas y prefabs de Unity, complementados con scripts C\# que encapsulan la lógica de juego. A nivel alto, se distinguen los siguientes componentes:

\begin{itemize}
    \item \textbf{Gestor de Juego (GameManager)}: coordina el flujo general (carga de escenas, inicio y fin de carrera).
    \item \textbf{Controlador de Vehículo}: maneja la entrada del usuario y traduce las acciones en movimiento del auto.
    \item \textbf{Gestor de Preguntas}: administra el banco de preguntas, selecciona ítems y evalúa respuestas.
    \item \textbf{Portales de Respuesta}: representan visualmente cada opción y aplican los efectos correspondientes.
    \item \textbf{UI Manager}: muestra textos, preguntas, opciones y mensajes de retroalimentación.
\end{itemize}

\subsection{Escenas y estructura básica}

El proyecto se compone de varias escenas principales:

\begin{itemize}
    \item \textbf{Pantalla de inicio}: portada del juego y acceso al menú principal.
    \item \textbf{Menú principal}: selección de modo de juego, vehículos y opciones.
    \item \textbf{Escena de carrera}: circuito de juego con vehículos, portales y HUD.
    \item \textbf{Pantalla de resultados}: resumen de tiempo, aciertos y desempeño.
\end{itemize}

% Ejemplo de lugar donde luego se puede insertar un diagrama
% \begin{figure}[h]
%     \centering
%     \includegraphics[width=0.8\textwidth]{diagrama_arquitectura.png}
%     \caption{Diagrama general de la arquitectura del videojuego.}
% \end{figure}

\section{Implementación y funcionalidades}

\subsection{Mecánica de conducción}

La conducción se basa en la lectura de entradas de teclado o mando (acelerar, frenar, girar) aplicadas sobre un controlador de física del vehículo. Se ajustan parámetros como aceleración máxima, fuerza de frenado y sensibilidad de giro para ofrecer una experiencia de manejo estable y apta para usuarios novatos.

\subsection{Sistema de preguntas y portales}

El sistema de preguntas utiliza una colección de ítems donde cada elemento incluye enunciado, alternativas y respuesta correcta. Al activarse una zona de pregunta, se muestra el enunciado en la interfaz y se habilitan portales asociados a cada alternativa; el vehículo atraviesa uno de ellos y la respuesta se evalúa de forma inmediata, modificando el estado del vehículo a “impulsado” o “penalizado”.

\subsection{Selección de vehículos y estadísticas}

La pantalla de selección de vehículos permite al jugador comparar varios autos con atributos como velocidad máxima, aceleración, capacidad de giro y frenado. Estos atributos se reflejan directamente en el comportamiento del controlador del vehículo dentro de la pista, permitiendo que el usuario elija el estilo de manejo que mejor se adapta a su estrategia.

\subsection{Manejo de errores y buenas prácticas}

En el desarrollo se consideró el manejo básico de errores y validaciones, por ejemplo:

\begin{itemize}
    \item Verificar que exista una pregunta válida antes de mostrarla.
    \item Evitar estados inconsistentes del vehículo (como aplicar impulso y penalización simultáneos).
    \item Registrar mensajes en la consola de Unity cuando se detectan configuraciones incompletas en la escena.
\end{itemize}

\section{Proceso de instalación y ejecución}

\subsection{Instalación para desarrolladores}

Para trabajar con el proyecto desde Unity:

\begin{enumerate}
    \item Clonar el repositorio desde GitHub:
    \begin{itemize}
        \item \texttt{https://github.com/rescobedoq/ryc}
    \end{itemize}
    \item Abrir la carpeta del proyecto en la versión recomendada de Unity (por ejemplo, una versión LTS 2022 compatible).
    \item Permitir que Unity importe escenas, scripts y \textit{assets} necesarios.
    \item Verificar que la escena principal de la carrera esté configurada en \textit{Build Settings} como escena de inicio.
    \item Ejecutar el juego en modo \textit{Play} para comprobar conducción, sistema de preguntas y HUD.
\end{enumerate}

\subsection{Instalación para usuarios finales}

Para ejecutar únicamente el videojuego:

\begin{enumerate}
    \item Descargar la versión compilada del juego (ejecutable y carpeta de datos) proporcionada por el equipo desarrollador.
    \item Descomprimir el paquete en una carpeta local con permisos de lectura y escritura.
    \item Ejecutar el archivo principal (por ejemplo, \texttt{RyC.exe}) desde el sistema operativo Windows.
\end{enumerate}

\section{Pruebas y validación}

\subsection{Estrategia de pruebas}

Se realizaron principalmente pruebas manuales de:

\begin{itemize}
    \item \textbf{Jugabilidad}: respuesta del vehículo a las entradas del usuario, dificultad del circuito y sensación de control.
    \item \textbf{Sistema de preguntas}: correcta aparición de preguntas, asociación con portales y aplicación de impulsos o penalizaciones.
    \item \textbf{Interfaz}: visibilidad de textos, mensajes de feedback y consistencia entre escenas.
\end{itemize}

\subsection{Funcionalidades verificadas}

Durante las pruebas se verificó que:

\begin{itemize}
    \item El jugador puede iniciar una partida desde el menú principal, seleccionando vehículo y modo de juego.
    \item Las preguntas se muestran en los puntos previstos de la pista y los portales responden correctamente a la elección del jugador.
    \item El HUD refleja información relevante (velocidad, mensajes de acierto/error, etc.).
\end{itemize}

\subsection{Trabajo futuro}

Entre las mejoras propuestas para versiones posteriores se encuentran:

\begin{itemize}
    \item Integrar música de fondo y efectos de sonido coherentes con la temática del juego.
    \item Ampliar el número de circuitos y bancos de preguntas.
    \item Incorporar un panel de administración para que los docentes gestionen preguntas desde una interfaz gráfica.
    \item Añadir tablas de puntuaciones y métricas de aprendizaje para seguimiento del progreso de los estudiantes.
\end{itemize}

\section{Conclusiones}

El videojuego educativo de carreras \textit{“Rápidos y Curiosos”} demuestra que es posible integrar mecánicas de conducción arcade con un sistema de evaluación formativa basado en preguntas y respuestas de manera coherente y motivadora. La relación directa entre el acierto en las respuestas y la ventaja en la carrera genera un vínculo claro entre aprendizaje y rendimiento dentro del juego.

Desde el punto de vista técnico, el uso de Unity, C\# y patrones de diseño como Singleton, State y Observer permitió construir una arquitectura modular y extensible, adecuada para incorporar nuevas funcionalidades en futuras iteraciones del proyecto. El trabajo colaborativo mediante GitHub y la documentación asociada (manual de usuario y memoria descriptiva) completan el ciclo de desarrollo esperado en el curso.

\end{document}
