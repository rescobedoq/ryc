\documentclass[12pt,a4paper]{article}
\usepackage[spanish]{babel}

\usepackage[utf8]{inputenc}
\usepackage[T1]{fontenc}
\usepackage{geometry}
\geometry{margin=2.2cm}
\usepackage{parskip}
\usepackage{graphicx}
\usepackage{hyperref}
\usepackage{float}

\begin{document}

\begin{titlepage}
    \centering
    {\large Universidad Nacional de San Agustín de Arequipa\\[0.2cm]
    Facultad de Ingeniería de Producción y Servicios\\[0.2cm]
    Escuela Profesional de Ingeniería de Sistemas\\[0.4cm]}
    {\large Curso: Tecnología de Objetos\\[0.2cm]}
    {\large Docente: Ing. Richart Smith Escobedo Quispe\\[1cm]}

    {\LARGE \textbf{Manual de Usuario}\\[0.5cm]}
    {\Large Videojuego Educativo de Carreras en Unity\\[0.2cm]
    \Large \textit{“Rápidos y Curiosos”}\\[1.2cm]}

    \vfill
    {\large Arequipa, 7 de diciembre de 2024}
\end{titlepage}

\tableofcontents
\newpage

\section{Acerca del videojuego}

El videojuego educativo de carreras \textit{“Rápidos y Curiosos”} es una aplicación interactiva desarrollada en Unity que combina la conducción de vehículos con un sistema de preguntas y respuestas orientado a reforzar contenidos académicos de forma lúdica. Durante la carrera, el jugador debe atravesar portales que representan posibles respuestas a una pregunta mostrada en pantalla.

El juego está pensado para ser utilizado en contextos educativos formales e informales, permitiendo que docentes y estudiantes integren dinámicas de competición y toma de decisiones rápidas como apoyo al proceso de enseñanza–aprendizaje.

\section{¿Quiénes deben leer este documento?}

Este manual está dirigido principalmente a:

\begin{itemize}
    \item \textbf{Jugadores}: estudiantes o usuarios finales que utilizarán el videojuego para aprender mientras compiten en las carreras.
    \item \textbf{Docentes o facilitadores}: profesores, tutores o responsables de cursos que deseen emplear el videojuego como recurso complementario en sus sesiones.
\end{itemize}

Los desarrolladores del proyecto también pueden utilizar este documento como referencia para comprender la experiencia de uso desde el punto de vista del usuario final.

\section{Requisitos del sistema}

\subsection{Requisitos de hardware}

Para una experiencia de juego fluida se recomienda:

\begin{itemize}
    \item Procesador equivalente a Intel Core i3 o superior.
    \item 4 GB de memoria RAM como mínimo (8 GB recomendado).
    \item Tarjeta gráfica con soporte para aceleración 3D básica.
    \item Al menos 500 MB de espacio libre en disco.
    \item Resolución de pantalla mínima de 1280x720 píxeles.
\end{itemize}

\subsection{Requisitos de software}

\begin{itemize}
    \item Sistema operativo Windows 10 o superior.
    \item Controladores de tarjeta gráfica actualizados.
    \item Permisos de ejecución de aplicaciones en la carpeta donde se instale el juego.
\end{itemize}

\section{Instalación y ejecución}

\subsection{Instalación en Windows}

\begin{enumerate}
    \item Descargar el paquete del videojuego \textit{“Rápidos y Curiosos”} (archivo comprimido o instalador) proporcionado por el equipo desarrollador.
    \item Descomprimir el contenido en una carpeta de su preferencia (por ejemplo, \texttt{C:\textbackslash Juegos\textbackslash RyC}).
    \item Verificar que la carpeta contenga el archivo ejecutable (por ejemplo, \texttt{RyC.exe}) y la carpeta de datos asociada.
    \item Hacer doble clic sobre el ejecutable para iniciar el juego.
\end{enumerate}

\subsection{Ejecución desde el entorno de desarrollo}

Si se cuenta con Unity y el código fuente del proyecto:

\begin{enumerate}
    \item Clonar el repositorio del proyecto desde GitHub:
    \begin{itemize}
        \item \texttt{https://github.com/rescobedoq/ryc}
    \end{itemize}
    \item Abrir el proyecto en Unity (versión LTS recomendada).
    \item Seleccionar la escena principal del juego y pulsar el botón \textit{Play} en el editor.
\end{enumerate}

\section{Pantallas y funcionalidades principales}

\subsection{Pantalla de inicio del videojuego}

\begin{figure}[h]
    \centering
    \includegraphics[width=0.8\textwidth]{img/pantalla_inicio}
    \caption{Pantalla de inicio del videojuego.}
\end{figure}

Al ejecutar el videojuego, se muestra una pantalla inicial con:

\begin{itemize}
    \item El título del juego y una breve descripción de la experiencia educativa.
    \item Botones para acceder al \textbf{menú principal}, a las \textbf{opciones} y, opcionalmente, a una sección de \textbf{créditos}.
\end{itemize}

Desde esta pantalla el usuario puede avanzar al menú principal para configurar su partida y comenzar a jugar.

\subsection{Menú principal}

El menú principal del videojuego contiene los siguientes bloques:

\begin{itemize}
    \item \textbf{Modo de juego}: selección entre modo un jugador y modo multijugador local.
    \item \textbf{Selección de vehículo}: acceso a la pantalla donde se elige el auto y se visualizan sus estadísticas.
    \item \textbf{Iniciar carrera}: botón para comenzar la partida con la configuración actual.
    \item \textbf{Salir}: para cerrar el juego y volver al sistema operativo.
\end{itemize}


Cada bloque está diseñado para ser fácilmente accesible mediante ratón.

\subsection{Iniciar una partida}

Para iniciar una partida típica, el usuario debe:

\begin{enumerate}
    \item Seleccionar el \textbf{modo de juego} (un jugador o multijugador local).
    
    \item Elegir el \textbf{vehículo} preferido para cada jugador, revisando sus atributos básicos.
    \item Confirmar la selección y hacer clic en el botón \textbf{Jugar}.

    \begin{figure}[H]
        \centering
        \includegraphics[width=0.7\textwidth]{img/eleccion_modo}
        \caption{Elección del modo de juego (1 o 2 jugadores.}
        \label{fig:paso3-fijo}
    \end{figure}

    \begin{figure}[H]
        \centering
        \includegraphics[width=0.7\textwidth]{img/1jugador_eleccion_carro_seleccionado}
        \caption{Selección de vehículo para 1 jugador.}
        \label{fig:paso3-fijo}
    \end{figure}

    \begin{figure}[H]
        \centering
        \includegraphics[width=0.7\textwidth]{img/2jugadores_eleccion_carro_seleccionados}
        \caption{Selección de vehículo para 2 jugadores.}
        \label{fig:paso3-fijo}
    \end{figure}
    
\end{enumerate}

Una vez cargada la escena de la pista, los vehículos aparecen en la línea de salida y comienza la cuenta regresiva para el inicio de la carrera.


Los controles pueden variar según la configuración del juego, pero en general incluyen:

\begin{itemize}
    \item \textbf{Acelerar y frenar}: teclas o botones asignados para controlar la velocidad del vehículo (w).
    \item \textbf{Giro a la izquierda/derecha}: teclas de dirección para cambiar la trayectoria (a / d).
    \item \textbf{Freno o derrape}: tecla dedicada para frenar de manera más brusca o controlar curvas cerradas (s).
\end{itemize}

    \begin{figure}[H]
        \centering
        \includegraphics[width=0.7\textwidth]{img/adelante}
        \caption{Tecla W para avanzar hacia adelante}
        \label{fig:paso3-fijo}
    \end{figure}

    \begin{figure}[H]
        \centering
        \includegraphics[width=0.7\textwidth]{img/derecha}
        \caption{Tecla D para direccionar a la derecha.}
        \label{fig:paso3-fijo}
    \end{figure}

    \begin{figure}[H]
        \centering
        \includegraphics[width=0.7\textwidth]{img/izquierda}
        \caption{Tecla A para direccionar a la izquierda.}
        \label{fig:paso3-fijo}
    \end{figure}


\subsection{Indicadores en pantalla (HUD)}

Durante la carrera, la interfaz de juego muestra información relevante:

\begin{itemize}
    \item \textbf{Velocidad actual} del vehículo.
    \item \textbf{Vuelta actual} y progreso en la pista (si corresponde).
    \item \textbf{Pregunta y opciones} cuando se activa un portal educativo.
    \item \textbf{Mensajes de estado}: notificaciones breves de respuesta correcta/incorrecta, impulso o penalización.
\end{itemize}

    \begin{figure}[H]
        \centering
        \includegraphics[width=0.7\textwidth]{img/stats}
        \caption{Estadísticas del vehículo}
        \label{fig:paso3-fijo}
    \end{figure}

    \begin{figure}[H]
        \centering
        \includegraphics[width=0.7\textwidth]{img/2jugadores_ganador}
        \caption{Visualización de información como número de vueltas, ganador, etc.}
        \label{fig:paso3-fijo}
    \end{figure}

    \begin{figure}[H]
        \centering
        \includegraphics[width=0.7\textwidth]{img/portales.jpg}
        \caption{Muestra de los portales con preguntas y sus 2 opciones}
        \label{fig:paso3-fijo}
    \end{figure}

Estos elementos permiten al jugador entender fácilmente su desempeño y el efecto de sus decisiones.

\subsection{Sistema de preguntas y portales}

En puntos específicos de la pista se encuentran portales asociados a respuestas de una pregunta educativa:

\begin{itemize}
    \item Cuando el vehículo se aproxima a una zona de pregunta, en pantalla aparece el \textbf{enunciado} y las \textbf{opciones} disponibles.
    \item Cada portal representa una respuesta posible; el jugador debe decidir rápidamente qué portal atravesar.
    \item Si la respuesta es correcta, el vehículo entra en un \textbf{estado de impulso} (aumento temporal de velocidad).
    \item Si la respuesta es incorrecta, el vehículo entra en un \textbf{estado de penalización} (disminución temporal de velocidad).
\end{itemize}

El sistema de preguntas se ejecuta de manera automática durante la carrera, y el jugador no necesita realizar acciones adicionales aparte de elegir el portal adecuado.

\section{Selección de vehículos y modos de juego}

\subsection{Selección de vehículos}

La pantalla de selección de vehículos permite:

\begin{itemize}
    \item Ver una \textbf{galería de autos} disponibles.
    \item Visualizar \textbf{estadísticas} como velocidad, aceleración, giro y frenado mediante barras o indicadores numéricos.
    \item Elegir un vehículo para cada jugador antes de iniciar la carrera.
\end{itemize}

Una vez seleccionado el vehículo, los atributos se aplican al controlador del auto dentro de la escena de juego.

\subsection{Modo un jugador}

En el modo un jugador:

\begin{itemize}
    \item Solo un vehículo es controlado por el usuario.
    \item El objetivo principal es completar el circuito, responder correctamente la mayor cantidad de preguntas y optimizar el tiempo total.
    \item La cámara sigue al único vehículo en pantalla.
\end{itemize}

Este modo es ideal para práctica individual y para que el jugador se familiarice con la mecánica educativa.

\subsection{Modo multijugador local}

En el modo multijugador local:

\begin{itemize}
    \item Dos vehículos (o más, según la versión) son controlados por distintos jugadores en el mismo dispositivo.
    \item La pantalla puede dividirse o gestionarse con varias cámaras, según la configuración implementada.
    \item Los jugadores compiten entre sí, respondiendo preguntas y aprovechando impulsos o evitando penalizaciones.
\end{itemize}

El objetivo es fomentar la competencia amistosa y comparar el rendimiento tanto en conducción como en aciertos educativos.

\section{Gestión de contenido educativo y progreso}

\subsection{Bancos de preguntas}

El videojuego utiliza bancos de preguntas clasificadas por tema o nivel de dificultad. Cada pregunta incluye:

\begin{itemize}
    \item Enunciado principal.
    \item Varias opciones de respuesta.
    \item Identificación de la opción correcta.
\end{itemize}

\subsection{Progreso y resultados}

Al finalizar una carrera, el juego puede mostrar un resumen con:

\begin{itemize}
    \item Tiempo total empleado.
    \item Número de respuestas correctas e incorrectas.
    \item Efecto global sobre el rendimiento en la pista.
\end{itemize}

Esta información permite al jugador y al docente evaluar el desempeño tanto en la parte lúdica como en la parte educativa.

\section{Casos de uso sugeridos}

\subsection{Sesión guiada en aula}

\begin{enumerate}
    \item El docente proyecta el juego y controla un vehículo o permite que un estudiante lo haga.
    \item El grupo discute en voz alta cada pregunta antes de elegir el portal.
    \item Se comentan las respuestas correctas e incorrectas y su impacto en la carrera.
\end{enumerate}

\subsection{Práctica individual}

\begin{enumerate}
    \item Cada estudiante juega en un equipo distinto o en distintos turnos.
    \item El objetivo es mejorar el tiempo total y el porcentaje de aciertos.
    \item El docente puede plantear metas mínimas de desempeño para reforzar contenidos.
\end{enumerate}

\section{Recomendaciones técnicas y de hardware}

Para una experiencia óptima se sugiere:

\begin{itemize}
    \item Cerrar otras aplicaciones pesadas mientras se ejecuta el juego.
    \item Ajustar la calidad gráfica desde el menú de opciones si el rendimiento es bajo.
\end{itemize}

\section{Preguntas frecuentes (FAQ)}

\subsection*{¿El juego guarda mis resultados?}

En esta versión, el juego puede mostrar los resultados al final de la partida, pero el almacenamiento persistente de estadísticas depende de la configuración implementada. Consulte con el docente o el equipo desarrollador sobre la versión utilizada.

\subsection*{¿Puedo cambiar el conjunto de preguntas?}

Sí, siempre que el docente o un desarrollador edite el banco de preguntas dentro del proyecto de Unity o en los archivos de configuración definidos para tal fin.

\subsection*{¿Qué ocurre si cierro el juego durante una carrera?}

La partida actual se perderá y deberá iniciarse una nueva carrera al volver a abrir el juego.

\section{Solución de problemas}

\subsection*{El juego no inicia}

\begin{itemize}
    \item Verifique que el archivo ejecutable no haya sido bloqueado por el antivirus.
    \item Asegúrese de que su sistema operativo cumple los requisitos mínimos.
    \item Intente ejecutar el juego como administrador.
\end{itemize}

\subsection*{El juego va muy lento}

\begin{itemize}
    \item Reduzca la calidad gráfica desde el menú de opciones, si está disponible.
    \item Cierre otras aplicaciones que consuman muchos recursos.
    \item Verifique que los controladores de la tarjeta gráfica estén actualizados.
\end{itemize}

\subsection*{No aparecen preguntas o portales}

\begin{itemize}
    \item Asegúrese de haber iniciado el modo de juego correcto.
    \item Si está utilizando una versión de desarrollo, revise con el equipo que la escena tenga configurados los disparadores de preguntas.
\end{itemize}

\section{Contacto y soporte}

Para reportar problemas, sugerir mejoras o solicitar soporte técnico se recomienda:

\begin{itemize}
    \item Revisar la documentación y el archivo \texttt{README.md} del repositorio de GitHub del proyecto.
    \item Comunicarse con el equipo desarrollador a través de los canales definidos en el curso (correo institucional).
\end{itemize}

% EJEMPLO DE CÓMO INSERTAR UNA IMAGEN (SUSTITUIR nombre_archivo.png)
% \begin{figure}[h]
%     \centering
%     \includegraphics[width=0.8\textwidth]{nombre_archivo.png}
%     \caption{Ejemplo de pantalla del menú principal del videojuego.}
% \end{figure}

\end{document}
